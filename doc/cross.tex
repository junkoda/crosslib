\documentclass[a4paper,11pt, fleqn]{article}
\usepackage{amsmath}
\usepackage{amssymb}
\usepackage[pdfborder={0 0 0}]{hyperref}
\usepackage{amsthm} % proof
\usepackage[T1]{fontenc}
\usepackage{bm}
\usepackage{xcolor}
\usepackage{sectsty} % change section colours
\usepackage{lipsum}
\usepackage{titlesec}
\usepackage{mathrsfs}

\newtheorem{definition}{Definition}
\newtheorem{lemma}{Lemma}
\newtheorem{proposition}{Proposition}
\newtheorem{formula}{Formula}[subsection]
\newtheorem{fs}{Formula}[section]


% Use Times fonts
\usepackage{newtxtext, newtxmath}
\usepackage[scaled]{helvet}

\newcommand{\dbar}{\textit{\dj}}
\newcommand{\Dbar}{\textit{\Dj}}


\setlength{\oddsidemargin}{0pt}   %%% left margin
\setlength{\textwidth}{159.2mm}
\setlength{\topmargin}{0mm}
\setlength{\headheight}{10mm}
\setlength{\headsep}{10mm}         %%% length between header and test
\setlength{\textheight}{219.2mm}

\setlength{\parindent}{0pt}


%
% Colours
%
\definecolor{Green}{rgb}{0.0, 0.5, 0.0}
\definecolor{Blue}{rgb}{0.16, 0.32, 0.75}
\definecolor{Red}{rgb}{0.82, 0.1, 0.26}


%
% Appearance difinitions
%
%\titlespacing*{\section}{0pt}{5.5ex plus 1ex minus .2ex}{4.3ex plus .2ex}
\titlespacing\section{0pt}{0pt}{24pt}

\title{cross}
\author{Jun Koda}
\date{}

\begin{document}
%
% Title page
%
\vspace{0.3 \paperheight}

\begin{center}
  {\Huge \textsc{Cross}}
\end{center}

\tableofcontents

\sectionfont{\sffamily\Huge\color{Green}\selectfont}
\subsectionfont{\sffamily\color{Green}\selectfont}
\subsubsectionfont{\sffamily\color{Green}\selectfont}
\paragraphfont{\sffamily\color{Green}\selectfont}

%
% Redshift-space distortions
%
\newpage

\section{Redshift-space distortions}

The redshift-space distortion $\bm{x} \mapsto \bm{s}$ is,
%
\begin{equation}
  \bm{s} = \bm{x} + \lambda u(\bm{x}) \, \hat{\bm{z}},
\end{equation}
where $\lambda \ge 0$ is a controll paramter; $\lambda = 0$ is the real space and $\lambda = 1$ is the redshift space.

$u(\bm{x})$ is the displacement field with plane-parallel along the $z$ axis.
\begin{equation}
  u(\bm{x}) = \frac{1}{aH} v_z(\bm{x}),
\end{equation}
where $\bm{v}(\bm{x})$ is the peculiar velocity.

\vspace{5mm}

Mass conservation,
%
\begin{equation}
  [1 + \delta(\bm{x})] \,d^3 x = [1 + \delta^s(\bm{s})] \,d^3 s,
\end{equation}
%
gives a formula for redshift-space density contrast,
%
\begin{equation}
  \label{eq:delta-s}
  \delta_D(\bm{k}) + \hat{\delta}^s(\bm{k}, \lambda)
  = \int \! d^3 x \, e^{-i\bm{k}\cdot\bm{x}} [1 + \delta(\bm{x})]
    e^{-i \lambda k_z u(\bm{x})},
\end{equation}
%
and redshift-space power spectrum,
%
\begin{equation}
  P^s(\bm{k}, \lambda) = \int \! d^3 r \, e^{-i\bm{k}\cdot\bm{r}}
  \Big\langle
      [1 + \delta(\bm{x})][1 + \delta(\bm{y})] \,
      e^{-i \lambda k_z [u(\bm{x}) - u(\bm{y})]}\Big\rangle,
\end{equation}
where $\delta_D$ is the Dirac delta function and $\bm{r} = \bm{x} - \bm{y}$.\\

The momentum field is defined as,
%
\begin{equation}
  p(\bm{x}) \equiv [1 + \delta(\bm{x})] \, u(\bm{x}).
\end{equation}

We define the density-momentum cross power spectrum as,
\begin{align}
  (2\pi)^3 \delta_D(\bm{k} - \bm{k}') P_{\delta p}(\bm{k}, \lambda)
  &= \mathrm{Im} \left\langle
  \hat{\delta}(\bm{k}, \lambda)^* \, \hat{p}(\bm{k}', \lambda)
  \right\rangle\\
%
  (2\pi)^3 \delta_D(\bm{k} - \bm{k}') P_{p \delta}(\bm{k}, \lambda)
  &= \mathrm{Im} \left\langle
  \hat{p}(\bm{k}', \lambda) \, \hat{\delta}(\bm{k}, \lambda)^*
  \right\rangle
\end{align}
The ensemble average is pure imaginary and antisymmetric,
%
\begin{equation}
  P_{\delta p}(\bm{k}, \lambda) = P_{p \delta}(-\bm{k}, \lambda)
  = -P_{p \delta}(\bm{k}, \lambda),
\end{equation}
from statistical parity invariance.\\

\clearpage
\subsection{Derivative}

\begin{equation}
  \frac{\partial P^s(\bm{k}, \lambda)}{\partial \lambda}
  = 2 k_z P_{p \delta}(\bm{k}, \lambda)
\end{equation}
    
\begin{equation}\begin{split}
  \frac{\partial P^s(\bm{k}, \lambda)}{\partial \lambda}
  &= -ik_z \int \! d^3r \, e^{-i\bm{k}\cdot\bm{r}} \left\langle
  [1 + \delta(\bm{x})] u(\bm{x}) \delta(\bm{y}) \right\rangle e^{-ik_z \Delta u}
  - [u(\bm{y}) \mbox{ term}]\\
  &= k_z P_{p\delta}(\bm{k}, \lambda) - k_z P_{\delta p}(\bm{k}, \lambda)\\
  &= 2k_z P_{p\delta}(\bm{k}, \lambda)
\end{split}\end{equation}

\end{document}
