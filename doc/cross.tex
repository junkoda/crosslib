\documentclass[a4paper,11pt, fleqn]{article}
\usepackage{amsmath}
\usepackage{amssymb}
\usepackage[pdfborder={0 0 0}]{hyperref}
\usepackage{amsthm} % proof
\usepackage[T1]{fontenc}
\usepackage{bm}
\usepackage{xcolor}
\usepackage{sectsty} % change section colours
\usepackage{lipsum}
\usepackage{titlesec}
\usepackage{mathrsfs}

\newtheorem{definition}{Definition}
\newtheorem{lemma}{Lemma}
\newtheorem{proposition}{Proposition}
\newtheorem{formula}{Formula}[subsection]
\newtheorem{fs}{Formula}[section]


% Use Times fonts
\usepackage{newtxtext, newtxmath}
\usepackage[scaled]{helvet}

\newcommand{\dbar}{\textit{\dj}}
\newcommand{\Dbar}{\textit{\Dj}}


\setlength{\oddsidemargin}{0pt}   %%% left margin
\setlength{\textwidth}{159.2mm}
\setlength{\topmargin}{0mm}
\setlength{\headheight}{10mm}
\setlength{\headsep}{10mm}         %%% length between header and test
\setlength{\textheight}{219.2mm}

\setlength{\parindent}{0pt}


%
% Colours
%
\definecolor{Green}{rgb}{0.0, 0.5, 0.0}
\definecolor{Blue}{rgb}{0.16, 0.32, 0.75}
\definecolor{Red}{rgb}{0.82, 0.1, 0.26}


%
% Appearance difinitions
%
%\titlespacing*{\section}{0pt}{5.5ex plus 1ex minus .2ex}{4.3ex plus .2ex}
\titlespacing\section{0pt}{0pt}{24pt}

\title{cross}
\author{Jun Koda}
\date{}

\begin{document}
%
% Title page
%
\vspace{0.3 \paperheight}

\begin{center}
  {\Huge \textsc{Cross}}
\end{center}

\tableofcontents

\sectionfont{\sffamily\Huge\color{Green}\selectfont}
\subsectionfont{\sffamily\color{Green}\selectfont}
\subsubsectionfont{\sffamily\color{Green}\selectfont}
\paragraphfont{\sffamily\color{Green}\selectfont}

%
% Redshift-space distortions
%
\newpage

\section{Redshift-space distortions}

The redshift-space distortion $\bm{x} \mapsto \bm{s}$ is,
%
\begin{equation}
  \bm{s} = \bm{x} + \lambda u(\bm{x}) \, \hat{\bm{z}},
\end{equation}
where $\lambda \ge 0$ is a control parameter; $\lambda = 0$ is the real space and $\lambda = 1$ is the redshift space.

$u(\bm{x})$ is the displacement field with plane-parallel line of
sight along the $z$ axis.
\begin{equation}
  u(\bm{x}) = \frac{1}{aH} v_z(\bm{x}),
\end{equation}
where $\bm{v}(\bm{x})$ is the peculiar velocity.

\vspace{5mm}
\subsection{Density}

Mass conservation,
%
\begin{equation}
  [1 + \delta(\bm{x})] \,d^3 x = [1 + \delta^s(\bm{s})] \,d^3 s,
\end{equation}
%
gives a formula for redshift-space density contrast,
%
\begin{equation}
  \label{eq:delta-s}
  \delta_D(\bm{k}) + \hat{\delta}^s(\bm{k}, \lambda)
  = \int \! d^3 x \, e^{-i\bm{k}\cdot\bm{x}} [1 + \delta(\bm{x})]
    e^{-i \lambda k_z u(\bm{x})},
\end{equation}
%
and redshift-space power spectrum,
%
\begin{equation}
  P^s(\bm{k}, \lambda) = \int \! d^3 r \, e^{-i\bm{k}\cdot\bm{r}}
  \Big\langle
      [1 + \delta(\bm{x})][1 + \delta(\bm{y})] \,
      e^{-i \lambda k_z [u(\bm{x}) - u(\bm{y})]} - 1\Big\rangle,
\end{equation}
where $\delta_D$ is the Dirac delta function and $\bm{r} = \bm{x} - \bm{y}$.
We also abbriviate,
\begin{equation}
  \Delta u \equiv u(\bm{x}) - u(\bm{y}).
\end{equation}

%
% Momentum
%
\vspace{5mm}
\subsection{Momentum}

The momentum field is defined as,
%
\begin{equation}
  p(\bm{x}) \equiv [1 + \delta(\bm{x})] \, u(\bm{x}).
\end{equation}
%
The ensamble average is zero, $\langle p(\bm{x}) \rangle = 0$.

We define the density-momentum cross power spectrum as,
\begin{align}
  (2\pi)^3 \delta_D(\bm{k} - \bm{k}') P_{\delta p}(\bm{k})
  &= \mathrm{Im} \left\langle
  \hat{\delta}(\bm{k})^* \, \hat{p}(\bm{k}')
  \right\rangle\\
%
  (2\pi)^3 \delta_D(\bm{k} - \bm{k}') P_{p \delta}(\bm{k})
  &= \mathrm{Im} \left\langle
  \hat{p}(\bm{k}) \, \hat{\delta}(\bm{k}')^*
  \right\rangle
\end{align}
The ensemble average is pure imaginary and antisymmetric,
%
\begin{equation}
  P_{\delta p}(\bm{k}) = P_{p \delta}(-\bm{k})
  = -P_{p \delta}(\bm{k}),
\end{equation}
from statistical parity invariance.\\

The redshift-space cross power spectrum is defined similarly for $\delta^s(\bm{k}, \lambda)$ and $p^s(\bm{k}, \lambda)$,
%
\begin{equation}
  p^s(\bm{x}) d^3 s = [1 + \delta^s(\bm{s})] \,u^s(\bm{s}) d^3 s
    = [1 + \delta(\bm{x}) ] \, u(\bm{x}) d^3 x,
\end{equation}
%
\begin{align}
  P^s_{10} &= P^s_{pd}(\bm{k}, \lambda) = \mathrm{Im}
  \int\! d^3 r \, e^{-i \bm{k}\cdot\bm{r}}
  \Big\langle
  [ 1 + \delta(\bm{x}) ] u(\bm{x}) \delta(\bm{y})
  e^{-i k_z \Delta u} \Big\rangle,\\
%
  P^s_{11} &= P^s_{pp}(\bm{k}, \lambda) = 
  \int\! d^3 r \, e^{-i \bm{k}\cdot\bm{r}}
  \Big\langle
  [ 1 + \delta(\bm{x}) ] u(\bm{x}) [ 1 + \delta(\bm{y}) ] u(\bm{y})
  e^{-i k_z \Delta u} \Big\rangle.  
\end{align}
%
Note that the magnitude of $u$ is not rescaled with $\lambda$, only the
magnitude of redshift space distortion is scaled.\\


The shot noise is the mass-weighted velocity dispersion,
\begin{equation}
  S_{pp} = \bar{n}^{-1} \langle u(\bm{x})^2 \rangle_\mathrm{mass}
        \sim \bar{n}^{-1} \frac{1}{N} \sum_i u_i^2
\end{equation}
which is summed over $N$ particels in a volume $V$, and
$\bar{n} = N/V$ is the mean number density.

%
% Energy
%
\vspace{5mm}
\subsection{Energy}

\begin{equation}
  e(\bm{x}) \equiv [ 1 + \delta(x) ] \, u(\bm{x})^2.
\end{equation}
%
The ensamble average is nonzero, $\langle e \rangle \neq 0 $; it is
the mass-weighted velocity dispersion,
%
\begin{equation}
  \langle e(\bm{x}) \rangle = \langle [ 1 + \delta(\bm{x})] u(\bm{x})^2 \rangle
  \equiv \langle u(x)^2 \rangle_\mathrm{mass} \sim \frac{1}{V}\sum_i u_i^2,
\end{equation}

\begin{equation}
  P^s_{20} = P^s_{e\delta}(\bm{k}, \lambda)
  = \int \! d^3r \, e^{-i\bm{k}\cdot\bm{r}} \Big\langle
    [ 1 + \delta(\bm{x}) ] u(\bm{x})^2 [ 1 + \delta(\bm{y}) ]
    e^{-ik_z \Delta u } \Big\rangle.
\end{equation}

The energy-density power spectrum containes a zero mode,
\begin{equation}
  P_{e\delta}(\bm{k}, \lambda)
  \sim (2\pi)^3 \delta_D(\bm{k}) \, \langle u(\bm{x})^2 \rangle.
\end{equation}

The shot noise is the same as the momentum-momentum power spectrum,
\begin{equation}
  S_{e\delta} = \bar{n}^{-1} \langle u(\bm{x})^2 \rangle_\mathrm{mass}
\end{equation}

%
% Derivatives
%
\vspace{5mm}
\subsection{Derivatives}

\begin{align}
  \frac{\partial P^s(\bm{k}, \lambda)}{\partial \lambda}
  &= 2 k_z P^s_{p \delta}(\bm{k}, \lambda)\\
  %
  \frac{\partial P_{p\delta}^s(\bm{k}, \lambda)}{\partial \lambda}
  &= k_z \left[ P^s_{pp}(\bm{k}, \lambda) - P^s_{e\delta}(\bm{k}, \lambda)
    \right].
\end{align}

\begin{proof}
\begin{equation}\begin{split}
  \frac{\partial P^s(\bm{k}, \lambda)}{\partial \lambda}
  &= -ik_z \int \! d^3r \, e^{-i\bm{k}\cdot\bm{r}} \left\langle
       [1 + \delta(\bm{x})] u(\bm{x}) \delta(\bm{y}) \right\rangle
       e^{-i\lambda k_z \Delta u}
       - [u(\bm{y}) \mbox{ term}]\\
  &= k_z P_{p\delta}(\bm{k}, \lambda) - k_z P_{\delta p}(\bm{k}, \lambda)\\
  &= 2k_z P_{p\delta}(\bm{k}, \lambda).
\end{split}\end{equation}

\begin{equation}\begin{split}
  i \frac{\partial P_{p\delta}^s(\bm{k}, \lambda)}{\partial \lambda}
  &= -ik_z \int\! d^3 r \, e^{-i\bm{k}\cdot\bm{r}} \Big\langle
  [1 + \delta(\bm{x})]u(\bm{x})^2 [ 1 + \delta(\bm{y}) ]
  - [1 + \delta(\bm{x})]u(\bm{x}) [ 1 + \delta(\bm{y}) ]
  \Big\rangle\\
  &= -ik_z \left[ P^s_{e\delta}(\bm{k}, \lambda) - P^s_{pp}(\bm{k}, \lambda)
    \right]
\end{split}\end{equation}
\end{proof}

\clearpage
%
% Uncorrerated velocity
%
\section{Uncorrelated velocity}

Suppose the velocity is an uncorrelated field,
\begin{equation}
  \langle u(\bm{x}) u(\bm{y}) \rangle = 0 \mbox{ for $\bm{x} \neq \bm{y}$},
\end{equation}
which is also uncorrelated with the density.
In such a case, the power spectrum is simply damped by the
characteristic function,
%
\begin{equation}
  \phi(\lambda) \equiv \left\langle e^{-i\lambda u(\bm{x})}\right\rangle,
\end{equation}
%
\begin{equation}
  P^s(\bm{k}, \lambda) = P(k) \left\langle e^{-i\lambda k_z u} \right\rangle^2
  = P(k) \phi(\lambda k_z)^2.
\end{equation}

The momentum-density cross power become negative when the derivative of the characteristic funtion become negative; there is no physics here,
\begin{equation}
  \frac{\partial P^s(\bm{k}, \lambda)}{\partial \lambda}
  = 2k_z P_{p\delta}(\bm{k}, \lambda)
  = P(k) \frac{\partial \phi(\lambda k_z)^2}{\partial \lambda}.
\end{equation}



\clearpage
\section{Linear theory}

\begin{equation}
  P_{\delta\delta}^{(1)}(k) \equiv \bar{P}(k)
\end{equation}
\begin{equation}
  P_{p\delta}^{(1)}(\bm{k}) = \mathrm{Im} \bar{P}_{u\delta}(\bm{k})
  = f\mu \bar{P}(k)/k.
\end{equation}

\end{document}
